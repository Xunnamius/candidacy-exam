\chapter{StrongBox I, II, III} \label{chp:strongbox}

In this section we describe the three primary components of the StrongBox
research project, and define a timeline of completion.

\section{Project Timeline}

These are dates we expect certain research milestones to be accomplished:

\begin{enumerate}
    \item \textbf{4/2019} Complete use case implementation and case studies.
    \item \textbf{5/2019} Literature review on FDE theory and formal
    organization.
    \item \textbf{5-6/2019} StrongBox II cipher switching draft paper is
    completed.
    \item \textbf{6/2019} Formal theoretical framework to model StrongBox
    security guarantees.
    \item \textbf{12/2019} Analysis of security model and guarantees versus
    AES-XTS and other schemes.
    \item \textbf{2/2020} StrongBox III (theoretical security framework) draft
    paper is completed.
\end{enumerate}

\section{StrongBox I: Confidentiality, Integrity, and Performance using Stream Ciphers}

StrongBox I is the original published work demonstrating how recent developments
in hardware coupled with core insights about filesystem behavior invalidate the
assumption that practically mitigating attacks against high performance stream
ciphers for FDE is infeasible.

With this research, it was shown that recent developments in mobile hardware
invalidate the assumption that stream ciphers are unsuitable for FDE, making it
possible to take advantage of fast stream ciphers. Modern mobile devices employ
solid-state storage with Flash Translation Layers (FTL), which operate similarly
to Log-structured File Systems (LFS). They also include trusted hardware such as
Trusted Execution Environments (TEEs) and secure storage areas. StrongBox I,
implemented with the ChaCha20 stream cipher, leveraged these two trends to
outperform dm-crypt, the de-facto Linux FDE endpoint.

\subsection{Completed Work}

\begin{itemize}
    \item \textbf{Published.} \textit{StrongBox: Confidentiality, Integrity, and
    Performance using Stream Ciphers for Full-Drive Encryption} was published in
    the Proceedings of the Twenty-Third International Conference on
    Architectural Support for Programming Languages and Operating Systems
    (ASPLOS'18).

    \item \textbf{Implemented.} A prototype of the StrongBox approach was
    implemented in C. The full 5000+ LoC source for StrongBox I is available
    open source (see \appref{availability}).

    \item \textbf{Patented.} The StrongBox approach has been patented by the
    University of Chicago, pending prosecution and non-provisional filing.

    \item \textbf{Inspired.} Using stream ciphers for Full-Drive Encryption
    motivated research in a similar vein: using stream ciphers with highly
    available systems to secure arbitrary resources on the internet (HASCHK).
\end{itemize}

\section{StrongBox II: Energy, Security, and Performance Tradeoffs with Cipher Switching}

With StrongBox II, we evaluate a new StrongBox implementation capable of using
ciphers beyond ChaCha20 and AES-CTR; specifically, the following profile 1
stream ciphers were added to StrongBox: Sosemanuk, HC-128, Rabbit, Salsa20,
Salsa12, and Salsa8, ChaCha12, ChaCha8, Freestyle, and ARM Neon SIMD accelerated
versions of ChaCha.

The eSTREAM portfolio ciphers fall into two profiles. Profile 1 contains stream
ciphers more suitable for software applications with high throughput
requirements. Profile 2 stream ciphers are particularly suitable for hardware
applications with restricted resources such as limited storage, gate count, or
power consumption. This research does not consider profile 2 stream ciphers
offered by the eSTREAM portfolio.

\subsection{Completed Work}

\begin{itemize}
    \item \textbf{Cipher abstraction and eSTREAM implementations.} With
    StrongBox II, we abstracted the cipher interface into an independent
    subsystem such that StrongBox II functions with any stream or block cipher,
    including the original ChaCha20. We also included an implementation of the
    Freestyle randomized output stream cipher for consideration with FDE,
    similarly implemented in C. The full source code for StrongBox II is
    available alongside the original (see \appref{availability}).

    \item \textbf{Cipher switching framework and tradeoff exploration.} We
    performed experiments to quantify the performance and security properties of
    the various ciphers we implemented. From that, we engineered a ``cipher
    switching'' abstraction allowing StrongBox to switch (offline) between any
    cipher in its library. Further, by studying the contention between
    StrongBox's observed \emph{energy} use, the \emph{security} of the StrongBox
    construction given a certain cipher configuration, and the
    \emph{performance} of the construction under a given workload, we modeled a
    navigable tradeoff space where manipulating magnitude in one dimension
    yields a non-linear change in another.

    \item \textbf{Cipher switching strategies.} We implemented five ``switching
    strategies'' based on our experimental observations. These strategies allow
    StrongBox to navigate the aforementioned tradeoff space both offline
    \emph{and online}.
\end{itemize}

\subsection{Current Work}

\begin{itemize}
    \item \textbf{4-5/2019}
        \begin{itemize}
            \item Complete experimental evaluation of (online) cipher switching
            strategies.
            \item Finish use case framework implementation and determine worthy
            use case implementations based on the aforementioned evaluation.
            \item Complete study and experimental evaluation of worthy use
            cases.
        \end{itemize}
    \item \textbf{5-6/2019}
        \begin{itemize}
            \item Collate research into paper: \textit{StrongBox II: A Study of
            Practical Tradeoffs Between Energy, Performance, and Security}.
        \end{itemize}
\end{itemize}

\subsection{Considerations}

At this point in the project, we have established a useful set of pareto curves,
the cipher switching strategy implementations are practical, and our use cases
are built on top of those switching strategies. Our biggest concern is accurate
and meaningful energy readings. Integrating the energy use telemetry (i.e.
Energymon) directly into the StrongBox core instead of at the test fixture level
along with other tweaks and checks should address this. However, even without a
robust set of metrics on energy use, the performance-security curve alone
motivates the switching strategies and, in turn, the use cases.

\section{StrongBox III: Theoretical Framework for Analyzing StrongBox FDE}

With the final component of this research, at minimum we want: (1) define a set of
formal FDE security notions in context, (2) define the cryptographic goals and
guarantees of StrongBox, (3) introduce a theoretical framework and models to
evaluate StrongBox, (4) potentially explore AES-XTS security guarantees versus
StrongBox guarantees.

\subsection{Current Work}

\begin{itemize}
    \item \textbf{5-6/2019}
        \begin{itemize}
            \item Complete review of literature on relevant theoretical
            frameworks that examine security within the constraints of FDE, e.g.
            efficient constructions that achieve security in context and subject
            to practical constraints (Rogaway, Fruhwirth, et al).
            \item Outline well-defined cryptographic goals StrongBox should
            achieve; outline guarantees any FDE scheme is expected to provide.
        \end{itemize}
    \item \textbf{6-12/2019}
        \begin{itemize}
            \item Introduce theoretical framework to formalize important
            security notions and evaluate StrongBox FDE, perhaps under IND-CPA,
            IND-CCA, ``time-to-bruteforce'' where relevant, or related
            definitions.
            \item Explore AES-XTS security guarantees versus StrongBox ciphers
            in an appropriate mode, perhaps including key management concerns
            (e.g. circular encryption) and StrongBox's inherent cryptographic
            agility.
        \end{itemize}
    \item \textbf{1-2/2020}
        \begin{itemize}
            \item Collate research into paper: \textit{A Formal Analysis of
            StrongBox FDE}.
        \end{itemize}
\end{itemize}
